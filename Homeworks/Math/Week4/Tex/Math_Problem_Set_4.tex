\documentclass[letterpaper,12pt]{article}
\usepackage{array}
\usepackage{threeparttable}
\usepackage{geometry}
\geometry{letterpaper,tmargin=1in,bmargin=1in,lmargin=1.25in,rmargin=1.25in}
\usepackage{fancyhdr,lastpage}
\pagestyle{fancy}
\lhead{}
\chead{}
\rhead{}
\lfoot{\footnotesize\textsl{OSM Lab, Summer 2017, Math PS \#4}}
\cfoot{}
\rfoot{\footnotesize\textsl{Page \thepage\ of \pageref{LastPage}}}
\renewcommand\headrulewidth{0pt}
\renewcommand\footrulewidth{0pt}
\usepackage[format=hang,font=normalsize,labelfont=bf]{caption}
\usepackage{amsmath}
\usepackage{amssymb}
\usepackage{amsthm}
\usepackage{natbib}
\usepackage{setspace}
\usepackage{float,color}
\usepackage[pdftex]{graphicx}
\usepackage{hyperref}
\hypersetup{colorlinks,linkcolor=red,urlcolor=blue,citecolor=red}
\theoremstyle{definition}
\newtheorem{theorem}{Theorem}
\newtheorem{acknowledgement}[theorem]{Acknowledgement}
\newtheorem{algorithm}[theorem]{Algorithm}
\newtheorem{axiom}[theorem]{Axiom}
\newtheorem{case}[theorem]{Case}
\newtheorem{claim}[theorem]{Claim}
\newtheorem{conclusion}[theorem]{Conclusion}
\newtheorem{condition}[theorem]{Condition}
\newtheorem{conjecture}[theorem]{Conjecture}
\newtheorem{corollary}[theorem]{Corollary}
\newtheorem{criterion}[theorem]{Criterion}
\newtheorem{definition}[theorem]{Definition}
\newtheorem{derivation}{Derivation} % Number derivations on their own
\newtheorem{example}[theorem]{Example}
\newtheorem{exercise}[theorem]{Exercise}
\newtheorem{lemma}[theorem]{Lemma}
\newtheorem{notation}[theorem]{Notation}
\newtheorem{problem}[theorem]{Problem}
\newtheorem{proposition}{Proposition} % Number propositions on their own
\newtheorem{remark}[theorem]{Remark}
\newtheorem{solution}[theorem]{Solution}
\newtheorem{summary}[theorem]{Summary}
%\numberwithin{equation}{section}
\bibliographystyle{aer}
\newcommand\ve{\varepsilon}
\newcommand\boldline{\arrayrulewidth{1pt}\hline}

\begin{document}

\begin{flushleft}
   \textbf{\large{Math Homework Week \#4, Continuous Optimization}} \\[5pt]
   OSM Lab, Eun-Seok Lee \\[5pt]

\end{flushleft}

\vspace{5mm}

\begin{enumerate}



	\item (6.1) \\
$min -e^{-w^Tx} \\
\phantom{----}s.t$ $-w^Tx \leq - w^TAw + w^TAy - a\\
\phantom{-----}y^Tw = w^Tx +b$


	\item (6.5) \\
$min -(0.07m + 0.05k) \\
\phantom{----}s.t$ $4m + 3k = 240,000\\
\phantom{------}2m + k = 6,000$

	\item (6.6) \\
$f_1(x,y) = 6xy + 4y^2 + y = 0\\
f_2(x, y) = 3x^2 + 8xy + x = 0\\  \\
Hessian =  \begin{bmatrix} 6y& 6x+8y+1 \\ 6x+8y+1& 8x \end{bmatrix}\\$
i) (0,0) -- saddle point because Hessian is indefinite.\\
ii) (0, -1/4) -- saddle point because Hessian is indefinite.\\
iii) (-1/3, 0) -- saddle point because Hessian is indefinite.\\
iv) (-1/9, -1/12) -- local maximum because Hessian is negative definite.\\


	\item (6.11) \\
$ x_1 = x_0 - \frac{f'(x_0)}{f''(x_0)}\\
f'(x) = 0\\
x = -\frac{b}{2a} $ is an unique maxizer. \\
$f''(x) = 2a > 0$ so, this is maximizer. \\
\\
Now, $x_1 = x_0 - (2ax_0 +b)\frac{1}{2a} = -\frac{b}{2a}$ \\
So, for any $x_0$, $x_1$ is an unique maxizer.
 


	\item (6.14) \\
I uploaded .py file on math/Week4.















\end{enumerate}

\vspace{25mm}

\bibliography{ProbStat_probset}

\end{document}
