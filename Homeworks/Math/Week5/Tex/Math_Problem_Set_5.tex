\documentclass[letterpaper,12pt]{article}
\usepackage{array}
\usepackage{threeparttable}
\usepackage{geometry}
\geometry{letterpaper,tmargin=1in,bmargin=1in,lmargin=1.25in,rmargin=1.25in}
\usepackage{fancyhdr,lastpage}
\pagestyle{fancy}
\lhead{}
\chead{}
\rhead{}
\lfoot{\footnotesize\textsl{OSM Lab, Summer 2017, Math PS \#4}}
\cfoot{}
\rfoot{\footnotesize\textsl{Page \thepage\ of \pageref{LastPage}}}
\renewcommand\headrulewidth{0pt}
\renewcommand\footrulewidth{0pt}
\usepackage[format=hang,font=normalsize,labelfont=bf]{caption}
\usepackage{amsmath}
\usepackage{amssymb}
\usepackage{amsthm}
\usepackage{natbib}
\usepackage{setspace}
\usepackage{float,color}
\usepackage[pdftex]{graphicx}
\usepackage{hyperref}
\hypersetup{colorlinks,linkcolor=red,urlcolor=blue,citecolor=red}
\theoremstyle{definition}
\newtheorem{theorem}{Theorem}
\newtheorem{acknowledgement}[theorem]{Acknowledgement}
\newtheorem{algorithm}[theorem]{Algorithm}
\newtheorem{axiom}[theorem]{Axiom}
\newtheorem{case}[theorem]{Case}
\newtheorem{claim}[theorem]{Claim}
\newtheorem{conclusion}[theorem]{Conclusion}
\newtheorem{condition}[theorem]{Condition}
\newtheorem{conjecture}[theorem]{Conjecture}
\newtheorem{corollary}[theorem]{Corollary}
\newtheorem{criterion}[theorem]{Criterion}
\newtheorem{definition}[theorem]{Definition}
\newtheorem{derivation}{Derivation} % Number derivations on their own
\newtheorem{example}[theorem]{Example}
\newtheorem{exercise}[theorem]{Exercise}
\newtheorem{lemma}[theorem]{Lemma}
\newtheorem{notation}[theorem]{Notation}
\newtheorem{problem}[theorem]{Problem}
\newtheorem{proposition}{Proposition} % Number propositions on their own
\newtheorem{remark}[theorem]{Remark}
\newtheorem{solution}[theorem]{Solution}
\newtheorem{summary}[theorem]{Summary}
%\numberwithin{equation}{section}
\bibliographystyle{aer}
\newcommand\ve{\varepsilon}
\newcommand\boldline{\arrayrulewidth{1pt}\hline}

\begin{document}

\begin{flushleft}
   \textbf{\large{Math Homework Week \#4, Continuous Optimization}} \\[5pt]
   OSM Lab, Eun-Seok Lee \\[5pt]

\end{flushleft}

\vspace{5mm}

\begin{enumerate}



	\item (7.1) \\
If S is a nonempty set, \\$\lambda_1x_1 + \cdots + \lambda_kx_k \in conv(S), x_i\in S, k\in N\\
 $where $ \lambda_i \geq 0$, and $\lambda_1 + \cdots + \lambda_k =1 $ \\ \\
It means that $\lambda x+(1-\lambda)y \in con(S) \ \ \forall \lambda $ s.t.$ 0 \leq \lambda \leq 1 $ \\
It is the definition of convex set.


	\item (7.2) \\
1) hyperplane is convex. \\
$P = \{ x\in V | \langle a,x\rangle \}$ where $ a \in V, a \neq 0, $ and $ b \in R \\ $\\
It means that, in the condition of $a_1x_1 + \cdots + a_nx_n = b $, \\
those $x_1, \cdots,  x_n$ has to be memebers of $P$.
\\
By the definition of convex, $\lambda x + (1-\lambda)y \in P \\
\because$ it is totally same expression to definition of hyperplane. \\ \\

2) half space is convex. \\
$H = \{ x \in V | \langle a,x\rangle \leq b \} \\$
It means that, in the condition of $a_1x_1 + \cdots + a_nx_n \leq b $, \\
those $x_1, \cdots,  x_n$ has to be memebers of $P$.
\\
By the definition of convex, $\lambda x + (1-\lambda)y \in P \\
\because$ it is totally same expression to definition of hyperplane. \\ \\

	\item (7.4) \\
(i) $\| x - p + p - y \|^2 = \| x - p\|^2 + \|p-y\|^2 + 2\|x-p\|\|p-y\|cos\theta  \\
 =  \| x - p \|^2 + \| p-y \|^2 + 2 \langle x-p, p-y\rangle
$\\
\\
(ii) $ \| x-y \|^2 \geq \|x-p\|^2 $ from (i) and assumption (7.14)$ \\
\therefore \| x-y\| > \|x-p \| \\
\because y \neq p \\
\therefore \|p-y\| >0 \\
$\\
(iii) $\| x - \lambda y - (1- \lambda)p \|^2 = \| x-p + \lambda (p-y)\|^2 \\
= \| x-p\|^2 + 2\| x-p\| \| \lambda(p-y)\|cos\theta  + \lambda^2 \| y-p\|^2 \\
=  \| x-p\|^2 + 2\lambda \langle x-p, p-y\rangle  + \lambda^2 \| y-p\|^2 \\
$ \\
(iv) $\frac{\| x-z \|^2 -\| x-p\|^2}{\lambda} = 2 \langle x-p, p-y \rangle + \lambda^2 \|y-p\|^2 $ from (iii)\\
By the definition of projection, LHS $\geq 0$ \\ Also, when $\lambda = 0$, $0\leq \langle x-p, p-y\rangle$




	\item (7.6) \\
$f(\lambda x + (1-\lambda) y \leq \lambda f(x) + (1 -\lambda) f(y) = c \\
$ RHS $= c$, Then, $x \in domain, y\in domain$, so, $(\lambda x + (1-\lambda)y) \in domain$\\
for satisfying convex function definition.


	\item (7.7) \\
$
\lambda_1 \theta_1f_1(x) + \lambda_1 (1-\theta_1)f_1(y) + \cdots + \lambda_n \theta_nf_n(x) + \lambda_n (1-\theta_n)f_n(x) \geq \lambda_1 f_1(\theta_1x + (1-\theta_1)y) + \cdots + \lambda_n f_n(\theta_nx + (1-\theta_n)y) \\
\therefore \lambda f(x) + (1-\lambda) f(y)  \geq f(\lambda x + (1-\lambda)y) \\
$

	\item (7.13) \\
If $f$ is not constant, $\exists x$ and $y $ s.t.$  f(x) \neq f(y)$\\ \\
1) Let x be the point of $max(f(x))$\\
If  $y_1 < x$ and $f(y_1) \neq f(x)$, $f(\lambda x + (1-\lambda)y_1) \leq \lambda f(x) + (1-\lambda) f(y_1)$ and $f(y_1) \leq f(x) $\\
If $y_2 > x$ and $ f(y_2) \neq f(x)$, $f(\lambda x + (1-\lambda)y_2) \leq \lambda f(x) + (1-\lambda) f(y_2)$ and $f(y_2) \leq f(x) $\\
Now, $\exists \lambda $ s.t. $ f(x) =\lambda f(y_1) + (1-\lambda) f(y_2) > \lambda f(y_1) + (1-\lambda) f(y_2), \\(\because f(x) \neq f(y_1), f(y_2))$\\ then, it is contradiction to the convex definition. \\

2) If there is no maximum point of the function, it has to be increasing function as x goes to infinity or it has to be decreasing function as x goes to infinity. if function increases and decreases, it has to be concave by the proof of "1)".  I will prove the case of the increasing function in a reason that the way of proving method is same.\\
(i) First, we can pick $x<y$ and $f(x) < f(y)$. If $ f(\lambda x + (1-\lambda)y) = \lambda f(x) + (1-\lambda)f(y)$, then, it should not be bounded because it is a line. \\
(ii) Second, we can pick $x<y$ and $f(x) < f(y)$. If $\lambda f(x) + (1-\lambda)f(y) <\lambda f(x) + (1-\lambda)f(y)$, then, $ f(t) \geq f(x) +(t-a)\frac{f(y) - f(x)}{y-x}$ s.t. $ t>y$ by convexity. Then, as $t$ goes infinity, it diverges and this is contradiction. \\
\\
I showed every possible cases. Therefore, Q.E.D..


	\item (7.20) \\
By the assumption, $\lambda f(x) + (1-\lambda) f(y) = f(\lambda x + (1-\lambda)y) \\
\therefore f(0) + \lambda f(x) = f(\lambda x) $ when $ y=0.$ \\
This is a general case and exactly affine form.

	\item (7.21) \\
$[\phi(f(x))]' = \phi ' (f(x))f'(x) = 0$ by the chain rule.\\
Then, $\phi(x)$ is strictly increasing, so $\phi '(f(x)) \neq 0$. \\
Therefore, $f'(x)=0$, and this is same FONC with ($min$  $f(x)$).










\end{enumerate}

\vspace{25mm}

\bibliography{ProbStat_probset}

\end{document}
