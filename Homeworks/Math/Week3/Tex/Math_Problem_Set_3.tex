\documentclass[letterpaper,12pt]{article}
\usepackage{array}
\usepackage{threeparttable}
\usepackage{geometry}
\geometry{letterpaper,tmargin=1in,bmargin=1in,lmargin=1.25in,rmargin=1.25in}
\usepackage{fancyhdr,lastpage}
\pagestyle{fancy}
\lhead{}
\chead{}
\rhead{}
\lfoot{\footnotesize\textsl{OSM Lab, Summer 2017, Math PS \#3}}
\cfoot{}
\rfoot{\footnotesize\textsl{Page \thepage\ of \pageref{LastPage}}}
\renewcommand\headrulewidth{0pt}
\renewcommand\footrulewidth{0pt}
\usepackage[format=hang,font=normalsize,labelfont=bf]{caption}
\usepackage{amsmath}
\usepackage{amssymb}
\usepackage{amsthm}
\usepackage{natbib}
\usepackage{setspace}
\usepackage{float,color}
\usepackage[pdftex]{graphicx}
\usepackage{hyperref}
\hypersetup{colorlinks,linkcolor=red,urlcolor=blue,citecolor=red}
\theoremstyle{definition}
\newtheorem{theorem}{Theorem}
\newtheorem{acknowledgement}[theorem]{Acknowledgement}
\newtheorem{algorithm}[theorem]{Algorithm}
\newtheorem{axiom}[theorem]{Axiom}
\newtheorem{case}[theorem]{Case}
\newtheorem{claim}[theorem]{Claim}
\newtheorem{conclusion}[theorem]{Conclusion}
\newtheorem{condition}[theorem]{Condition}
\newtheorem{conjecture}[theorem]{Conjecture}
\newtheorem{corollary}[theorem]{Corollary}
\newtheorem{criterion}[theorem]{Criterion}
\newtheorem{definition}[theorem]{Definition}
\newtheorem{derivation}{Derivation} % Number derivations on their own
\newtheorem{example}[theorem]{Example}
\newtheorem{exercise}[theorem]{Exercise}
\newtheorem{lemma}[theorem]{Lemma}
\newtheorem{notation}[theorem]{Notation}
\newtheorem{problem}[theorem]{Problem}
\newtheorem{proposition}{Proposition} % Number propositions on their own
\newtheorem{remark}[theorem]{Remark}
\newtheorem{solution}[theorem]{Solution}
\newtheorem{summary}[theorem]{Summary}
%\numberwithin{equation}{section}
\bibliographystyle{aer}
\newcommand\ve{\varepsilon}
\newcommand\boldline{\arrayrulewidth{1pt}\hline}

\begin{document}

\begin{flushleft}
   \textbf{\large{Math Homework Week \#3, Spectral Theory}} \\[5pt]
   OSM Lab, Eun-Seok Lee \\[5pt]

\end{flushleft}

\vspace{5mm}

\begin{enumerate}

	\item (4.2) 

$p = ax^2 + bx + c \\ \\
D(p)(x)=2ax+b\\ \\
 \therefore \begin{bmatrix}
0&1&0\\0&0&2\\0&0&0\end{bmatrix} \begin{bmatrix}
c\\b\\a\end{bmatrix} = \begin{bmatrix}b\\2a\\0\end{bmatrix}$  \\ \\ \\
$adjoint = \begin{bmatrix}
0&1&0\\0&0&2\\0&0&0\end{bmatrix}^T$ \\

$det\begin{bmatrix} - \lambda & 1 & 0\\ 0 & -\lambda & 2\\ 0 & 0 &-\lambda\end{bmatrix} = -\lambda^3=0 \\ \\ \therefore \lambda = 0$ \\
So, algebraic multiplicities: 3 \\
geometric multiplicities: 3 - 2(rank) = 1 


	\item (4.4) \\
$(i) A=\begin{bmatrix} a&b \\c&d\end{bmatrix} A^H=\begin{bmatrix}\bar{a} & \bar{b} \\ \bar{c} &\bar{d} \end{bmatrix}$ \\
$\lambda-(a+d)\lambda+(ad-bc)=0\\
b=\bar{c}, c=\bar{b}, a=\bar{a}, d=\bar{d}\\
\lambda^2-(a+d)\lambda+(ad-b^2)=0\\
(a-d)^2+4c\bar{c} \geq 0 \\ \\
(ii) A=\begin{bmatrix} a&b \\c&d\end{bmatrix} A^H=\begin{bmatrix}\bar{a} & \bar{b} \\ \bar{c} &\bar{d} \end{bmatrix}\\
b=-\bar{c}, c=-\bar{b}, a=-\bar{a}, d=-\bar{d}\\
(a-d)^2 -4b\bar{b} < 0 \ (\because a, d = 0$ or imaginary numbers.$)$

	\item (4.6) \\
Let diagonal entries of an upper-triangular matrix $A$ be $d_1, d_2, d_3, \cdots, d_n$. Then, $(A-\lambda I)$ is also upper triangular matrix. Now, diagonal entries are $d_i-\lambda$. $|\pi_i(d_i-\lambda)| = 0$ and $\lambda$ is $d_i$ here. For lower triangular matrix, the proof is almost same.


	\item (4.8) \\
(i) In terms of linear combination, it is linear independent. For example, when we calculate determinent of wronskian, it is not zero. \\ \\


(ii)
$D=\begin{bmatrix} 0&1&0&0 \\ -1&0&0&0 \\ 0&0&0&2 \\ 0&0&-2&0 \end{bmatrix}\begin{bmatrix} sinx \\ cosx \\sin2x \\cos2x\end{bmatrix}$
\\
\\
(iii)
$span[(1,0,0,0),(0,1,0,0)]$ and $span[(0,0,1,0), (0,0,0,1)]$

	\item (4. 13) \\
$P = \begin{bmatrix} 1&1 \\ 0.5&-1 \end{bmatrix}\\
Then, P^{-1}AP=\frac{1}{detp} \begin{bmatrix} -15&0 \\ 0&-6 \end{bmatrix}\\$
I calculated $P^{-1}AP$ by general $P=\begin{bmatrix} a&b \\ c&d \end{bmatrix}$, then input numbers.



	\item (4.15) \\
$A=P^{-1}BP$, then $A^k= P^{-1}B^kP\\$
If A is semisimple, then it is diagonalizable. By Thm 4.3.7 proof, $A=P^{-1}DP where D=diag(\lambda_1,\lambda_2, \cdots, \lambda_n)\\$
Then, 
$A^n= P^{-1}D^nP$, so $D^2=diag(\lambda_1^{n}, \cdots, \lambda_n^{n})$\\
Then, $(a_0+a_1A+ \cdots +a_nA^n)x=(a_0+\lambda + \lambda^2 + \cdots + \lambda^n)x\\$
So, $f(A)x = f(\lambda)x$





	\item (4.16) \\
(i) $\lim_{n\rightarrow\infty} A^n $ with respect to the 1-norm. \\
$P^{-1}AP=D$ where $P=\begin{bmatrix} 2&1 \\ 1&-2 \end{bmatrix}$ from eigen vectors.
So, we can easily calculate the answer $\begin{bmatrix} 2/3&2/3 \\ 1/3& 1/3\end{bmatrix}$ \\

(ii) In any form of norms, it has to be same in a reason that we calculate before taking norm.\\ \\
(iii) $f(x) = 3+5x+x^3 \\ \therefore f(\lambda_1)=9$ and $f(\lambda_2)=5.064$




	\item (4.18) \\
From $(Ax)=(\lambda x)$, $(Ax)^T=x^TA=(\lambda x)^T = x^T\lambda = \lambda x^T$




	\item (4.20) \\
If A is Hermitian and orthonormally similar to B, then, B is also Hermitian.\\
$\Rightarrow B=U^HAU$ where $U$ is orthonormal. \\$ B^H =U^HA^HU=U^HAU=B$ by $A$ is Hermitian.





	\item (4.24) \\
(i) $A=A^H \
\rho(x) = \frac{<x,Ax>}{\|x\|_2}=\frac{x^HAx}{\|x\|_2} = \frac{<x,\lambda x>}{\|x\|_2}=\frac{x^H\lambda x}{\|x\|_2} =\lambda = \frac{<A^Hx,x>}{\|x\|_2}=\frac{\bar{\lambda}x^Hx}{\|x\|_2} = \bar{\lambda}\\
A=-A^H \\$
$\rho(x) = \frac{<x,Ax>}{\|x\|_2}=\frac{<x,\lambda x>}{\| x\|_2}=\frac{\lambda x^Hx}{\| x\|_2} = \lambda = \frac{<A^Hx, x>}{\| x\|_2}=-\frac{<Ax, x>}{\|x\|_2}=-\frac{\bar{\lambda}x^Hx}{\|x\|_2}=-\bar{\lambda}
$



	\item (4.25) \\
(i) $(x_1x_1^H + \cdots +x_nx_n^H)(a_1x_1 + a_2x_2 + \cdots + a_nx_n) = (a_1x_1 + \cdots + a_nb_n)\\
\because x_j^Hx_j = 1 $  and by orthogonality.\\
By the way, $ (a_1x_1 + \cdots + a_nb_n)$ is arbitrary linear combination, so, \\$(x_1x_1^H + \cdots +x_nx_n^H)$ has to be $I$.
\\
\\
(ii) $AI = A(x_1x_1^H + \cdots +x_nx_n^H)\\
\therefore A = Ax_1x_1^H + \cdots +Ax_nx_n^H = \lambda x_1x_1^H + \cdots + \lambda x_nx_n^H
$

	\item (4.27) \\
let $x_i=(0, \cdots, 1+a*i, \cdots, 0)$.\\
Then, $x_i^HAx_I=((1+a*i)^Ha_{ii}(1+a*i)>0\\
\therefore a_{ii}>0 \\
(\because A$ is hermitian, so $A=A^H)$\\
Therefore, $a_{ii}$ is real.



 	\item (4.28) \\
We already know that trace is one of inner product in Chapter 3. So, it has to satisfy Cauchy-Schwartz ineqaulity.\\
So, $ tr(AB) \leq \sqrt{tr(A^2)tr(B^2)} = \sqrt{\sum (\lambda_i^A)^2\sum (\lambda_i^B)^2} \leq \sqrt{(\sum \lambda_i^A)^2(\sum \lambda_i^B)^2} = tr(A)tr(B)
$



 	\item (4.31) \\
(i) $\|A\|_2 = sup\frac{\| Ax\|_2}{\| x\|_2} =sup\frac{\| U\Sigma V^Hx\|_2}{\| x\|_2} =sup\frac{\| \Sigma V^Hx\|_2}{\| x\|_2} =sup\frac{\| \Sigma V^Hx\|_2}{\| V^Hx\|_2} = sup\frac{(\sum|\sigma_i V^Hx|^2 )^{0.5}}{(\sum|V^Hx|^2 )^{0.5}} \\$
(ii) By the method of (i), now $\frac{1}{\sigma_n}=\|A^{-1}\|_2 $ (It is the largest here.)$
$ \\
(iii) $\| A^H\|_2^2=\sigma_1^2=\| A^T\|_2^2 = \sigma_1^2 = \| A^HA\|_2 = \sigma_1^2 = (\| A\|_2)^2\\$
(iv) $\| UAV\|_2 = \| UU\Sigma V^HV\|_2 = \| \Sigma\|_2 = \| A\|_2$





 	\item (4.32) \\
(i) $\| UAV\|_F =  \| AV\|_F = \| U\Sigma V^HV\|_F = \| \Sigma \|_F = \| V \Sigma \|_F = \| \Sigma^H V^H \|_F \\(\because $ trace property)\\
$= \| U \Sigma V^H\|_F = \| A \|_F
$\\
(ii) $\| \Sigma \|_F = \sqrt{tr(\Sigma^H\Sigma)} = \sqrt{\sigma_1^2 + \cdots + \sigma_n^2}$



 	\item (4.33) \\
$\| A \|_2 = \| \Sigma \|_2 = sup|y^H\Sigma x| = \sigma_1
$\\ I already showed $\| A \|_2 = \| \Sigma \|_2 $  above.


 	\item (4.36) \\
$A = \begin{bmatrix} 1&2\\0&3  \end{bmatrix}\\
\lambda^2-4\lambda+3=0 \Rightarrow \lambda = 1$ or $3.$
From $A^HA=\begin{bmatrix} 1&0\\2&3  \end{bmatrix} \begin{bmatrix} 1&2\\0&3  \end{bmatrix}$, \\$\sigma_1 \approx 3.6503$, $\sigma_2\approx 0.8219
$




 	\item (4.38) \\
(i) $AA^+A = U_1\Sigma_1V_1^HV_1\Sigma_1^{-1}U_1^HA = U_1U_1^HA = U_1U_1^HU_1\Sigma_1V_1^H = U_1\Sigma_1V_1^H =A \\$
(ii) $A^+AA^+=V_1\Sigma^{-1}U_1^H = A^+\\$
(iii) $(AA^+)^H=U_1U_1^H=AA^+$ from (i) we can easily know it.\\
(iv) $ (A^+A)^H = V_1V_1^H = V_1\Sigma^{-1}U_1^HU_1\Sigma_1V_1^H = A^+A $\\
(v) From (iii),\\
From orthogonal projection, $A(A^HA)^{-1}A^H$, we can know that $A(A^HA)^{-1}A^HA = AA^+A = A$ \\
(vi) From (iv),\\
From orthogonal projection, $A^H(AA^H)^{-1}A$, we can know that $AA^H(AA^H)^{-1}A = AA^+A = A\\
$






















\end{enumerate}

\vspace{25mm}

\bibliography{ProbStat_probset}

\end{document}
