\documentclass[letterpaper,12pt]{article}
\usepackage{array}
\usepackage{threeparttable}
\usepackage{geometry}
\geometry{letterpaper,tmargin=1in,bmargin=1in,lmargin=1.25in,rmargin=1.25in}
\usepackage{fancyhdr,lastpage}
\pagestyle{fancy}
\lhead{}
\chead{}
\rhead{}
\lfoot{\footnotesize\textsl{OSM Lab, Summer 2017, Econ PS \#3}}
\cfoot{}
\rfoot{\footnotesize\textsl{Page \thepage\ of \pageref{LastPage}}}
\renewcommand\headrulewidth{0pt}
\renewcommand\footrulewidth{0pt}
\usepackage[format=hang,font=normalsize,labelfont=bf]{caption}
\usepackage{amsmath}
\usepackage{amssymb}
\usepackage{amsthm}
\usepackage{natbib}
\usepackage{setspace}
\usepackage{float,color}
\usepackage[pdftex]{graphicx}
\usepackage{hyperref}
\hypersetup{colorlinks,linkcolor=red,urlcolor=blue,citecolor=red}
\theoremstyle{definition}
\newtheorem{theorem}{Theorem}
\newtheorem{acknowledgement}[theorem]{Acknowledgement}
\newtheorem{algorithm}[theorem]{Algorithm}
\newtheorem{axiom}[theorem]{Axiom}
\newtheorem{case}[theorem]{Case}
\newtheorem{claim}[theorem]{Claim}
\newtheorem{conclusion}[theorem]{Conclusion}
\newtheorem{condition}[theorem]{Condition}
\newtheorem{conjecture}[theorem]{Conjecture}
\newtheorem{corollary}[theorem]{Corollary}
\newtheorem{criterion}[theorem]{Criterion}
\newtheorem{definition}[theorem]{Definition}
\newtheorem{derivation}{Derivation} % Number derivations on their own
\newtheorem{example}[theorem]{Example}
\newtheorem{exercise}[theorem]{Exercise}
\newtheorem{lemma}[theorem]{Lemma}
\newtheorem{notation}[theorem]{Notation}

\newtheorem{problem}[theorem]{Problem}
\newtheorem{proposition}{Proposition} % Number propositions on their own
\newtheorem{remark}[theorem]{Remark}
\newtheorem{solution}[theorem]{Solution}
\newtheorem{summary}[theorem]{Summary}
%\numberwithin{equation}{section}
\bibliographystyle{aer}
\newcommand\ve{\varepsilon}
\newcommand\boldline{\arrayrulewidth{1pt}\hline}

\begin{document}

\begin{flushleft}
   \textbf{\large{Econ Homework Week \#3, Firm Dynamics}} \\[5pt]
   OSM Lab, Eun-Seok Lee \\[5pt]

\end{flushleft}

\vspace{5mm}

\begin{enumerate}



	\item I modified given codes little bit, and applied "z\_grid" and "pi" to Value function. It looks reasonable. With higher z\_grid, it has higher value function.


	\item I think that the firms invest specific amount of k in early period in a reason that adjustment cost will increase as k increases fast rather than the previous cost function. Therefore, investment rate will decrease more faster. I set the dense as 1 to control "no investment policy."


	\item If I make policty function codes more efficient, it might be time-saving. But here, PFI takes longer time than VFI. (Actually, it is very slow...) The reason why I think PFI might be faster is that iteration time of PFI is shorter than VFI's when I execute the codes. When I see the result, the functions made from both of them look very similar. Because of processing time, I set weak tolerance and dense (1e-2 and 2). If I set those stonger, I expect that its shape will look more similar. I could have calculated the function for $k_{t+1}$ with repect to other variables. But, I would like to depend on computer rather than hand calculating.


	\item The result for the variables of interest is as follows.\\ \\
-- converged wage is 1.04695061\\
-- wage = 1.04695061\\
-- real interest rate = 0.04166666 (This is fixed by $(\frac{1}{\beta}-1)$)\\
-- Labor demand = 0.482755926825 \\
-- Labor supply = 0.485215909551 \\ \\
Around size "1" of capital, density was higher. In this problem, I attached the picture of distribution in the next page.












\end{enumerate}

\vspace{25mm}

\bibliography{ProbStat_probset}

\end{document}
