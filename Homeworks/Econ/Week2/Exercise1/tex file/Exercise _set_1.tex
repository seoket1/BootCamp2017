\documentclass[letterpaper,12pt]{article}
\usepackage{array}
\usepackage{threeparttable}
\usepackage{geometry}
\geometry{letterpaper,tmargin=1in,bmargin=1in,lmargin=1.25in,rmargin=1.25in}
\usepackage{fancyhdr,lastpage}
\pagestyle{fancy}
\lhead{}
\chead{}
\rhead{}
\lfoot{\footnotesize\textsl{OSM Lab, Summer 2017, Econ PS \#1}}
\cfoot{}
\rfoot{\footnotesize\textsl{Page \thepage\ of \pageref{LastPage}}}
\renewcommand\headrulewidth{0pt}
\renewcommand\footrulewidth{0pt}
\usepackage[format=hang,font=normalsize,labelfont=bf]{caption}
\usepackage{amsmath}
\usepackage{amssymb}
\usepackage{amsthm}
\usepackage{natbib}
\usepackage{setspace}
\usepackage{float,color}
\usepackage[pdftex]{graphicx}
\usepackage{hyperref}
\hypersetup{colorlinks,linkcolor=red,urlcolor=blue,citecolor=red}
\theoremstyle{definition}
\newtheorem{theorem}{Theorem}
\newtheorem{acknowledgement}[theorem]{Acknowledgement}
\newtheorem{algorithm}[theorem]{Algorithm}
\newtheorem{axiom}[theorem]{Axiom}
\newtheorem{case}[theorem]{Case}
\newtheorem{claim}[theorem]{Claim}
\newtheorem{conclusion}[theorem]{Conclusion}
\newtheorem{condition}[theorem]{Condition}
\newtheorem{conjecture}[theorem]{Conjecture}
\newtheorem{corollary}[theorem]{Corollary}
\newtheorem{criterion}[theorem]{Criterion}
\newtheorem{definition}[theorem]{Definition}
\newtheorem{derivation}{Derivation} % Number derivations on their own
\newtheorem{example}[theorem]{Example}
\newtheorem{exercise}[theorem]{Exercise}
\newtheorem{lemma}[theorem]{Lemma}
\newtheorem{notation}[theorem]{Notation}
\newtheorem{problem}[theorem]{Problem}
\newtheorem{proposition}{Proposition} % Number propositions on their own
\newtheorem{remark}[theorem]{Remark}
\newtheorem{solution}[theorem]{Solution}
\newtheorem{summary}[theorem]{Summary}
%\numberwithin{equation}{section}
\bibliographystyle{aer}
\newcommand\ve{\varepsilon}
\newcommand\boldline{\arrayrulewidth{1pt}\hline}

\usepackage{ amssymb }
\usepackage{graphicx}
\DeclareGraphicsExtensions{.pdf,.png,.jpg}





\begin{document}

\begin{flushleft}
   \textbf{\large{Excercise Set1 for Economics}} \\[5pt]
   OSM Lab, Eun-Seok Lee \\[5pt]
 
\end{flushleft}

\vspace{5mm}

\begin{enumerate}

       
	\item  (Please see py file) After computing both successive approximation and matrix algebra, the results were very similar. I set tolerance rate as "1e-7" but if I take
		 higher tolerance rate, it would become more similar. The successive approximation result is $[-0.89552246, 13.34328364, 45.64179143]^{T}$
		and matrix algebra result is $[-0.89552239, 13.34328358, 45.64179104]^{T}$.

	\item Claim: by Banah's fixed point theorem, it has an unique solution.\\
		 Metric Space is complete, so all I have to do is to prove T is a contraction mapping.\\
		$\rho(Tx,Ty) = |\beta \sum max(w_{k}, x)p_{k} - \beta \sum max(w_{k}, y)p_{k}| $ \\
				  $\leq \beta \sum |max(w_{k}, x)|p_{k} - \beta \sum |max(w_{k}, y)|p_{k}$ \\
				  $\leq \beta |x-y|$ \\
				  $= \beta\rho(x,y)$ \\ \\
                                  $\therefore \rho(Tx,Ty) \leq \beta\rho(x,y)$ \\ \\
		If this problem satisfies fixed point property, I can use successive approximation. In the next problem, I used it, and got a solution for reservation wage.

   		
   	\item (Please see py file) The reservation wage increases, and this coincides with my intuition; as compensation increases, the incentive for work may decrease.

\end{enumerate}

\vspace{25mm}



\end{document}
